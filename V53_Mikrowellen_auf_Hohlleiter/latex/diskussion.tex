\section{Diskussion}
\label{sec:Diskussion}

\autoref{fig:mode_plot} zeigt gut, dass die aufgenommenen Messwerte nicht ganz perfekt sind.
Die angeblichen Maxima der Moden sind nicht ganz mittig, wodurch die Fit Funktion eine Art neues Maximum erzeugt.
Dabei ist die Mode bei etwa $\SI{220}{\volt}$ die beste von den dreien.

Ein erstes Indiz für die Genuigkeit der Messung war die erneute Bestimmung der Frequenz.
Diese war als $f_\text{theorie} = \SI{9027}{\mega\hertz}$ eingestellt worden. 
Der in der Auswertung ausgerechnete Wert ist $f_\text{gemessen} = \SI{8872}{\hertz}$.
Die Abweichung von beiden ist somit $\Delta f = 1.75 \%$.

Es ist schwer die Dämpfungskurve richtig zu bewerten.
Die gemessenen Werte weichen, wie bereits beschrieben, um einen festen Wert von der Theoriekurve ab.
Wir haben also vermutlich einen systematischen Fehler beim Einstellen der Nulllinie gemacht.
Die Messwerte passen sich nämlich nach der Verschiebung gut dem Fit an.
Schlussendlich war die Messung an sich also erfolgreich.

Bei der Berechnung des Stehwellenverhältnisses gab es drei verschiedene Wege.
Vergleichen kann man dabei die SWRs bei denen die Tiefe des Gleitschraubentransformators $\SI{9}{\milli\meter}$ beträgt.
Für die drei Messergebnisse bekamen wir 
\begin{align*}
    S_1 &= \num{10.00} \\
    S_2 &= \num{8.987} \\
    S_3 &= \num{14.125} \, .
\end{align*}
$S_1$ war allerdings sehr schwer abzulesen. 
Wir mussten die Skala des SWR-Meters daran anpassen, da der Ausschlag immer maximal war.
Eigentlich sind die letzten beiden Methoden dafür geeignet Stehwellenverhältnisse bei großer Tiefe des Gleitschraubentransformators zu messen.
Hier liegen aber alle Werte in etwa um $S_1$ verteilt. 
Es ist also möglich, dass wir beim Ablesen von $S_1$ einen Fehler gemacht haben.