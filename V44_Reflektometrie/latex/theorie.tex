\section{Theorie}
\label{sec:Theorie}

\subsection{Reflexion und Brechungsindex}
\label{theo1}

Formel für n, Fresnel Gleichungen und so was

Trifft Röntgenstrahlung aus dem Vakuum auf ein Medium mit einem Brechungsindex $n \neq 1$ ist der Brechungsindex gegeben als
\begin{equation}
    n = 1 −  \delta + i * \beta.
    \label{eq:index}
\end{equation}
Dabei ist $\delta$ eine kleine Korrektur $\delta = 10^(-6)$ und $\beta$ die Absorption.



\subsection{Mehrschichtsysteme}
\label{theo2}

Parratt-Algorithmus, Kiessig Oszillationen 

\subsection{Rauigkeitskorrektur}
\label{theo3}

\subsection{Funktionsweise der Geräte im Versuch}
\label{theo4}
 
Kurze Beschreibung, vllt noch ne Skizze?