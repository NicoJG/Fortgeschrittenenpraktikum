\section{Diskussion}
\label{sec:Diskussion}

Alle durchgeführten Scans zur Justierung entsprechen unseren Erwartungen.
Auffällig ist jedoch, dass sich die Geometriewinkel
\begin{align*}
    \alpha_\text{g,gemessen} &= \SI{0.70}{\degree} \\
    \alpha_\text{g,Theorie} &= \SI{0.63}{\degree}
\end{align*}
so stark voneinander unterscheiden. 
Eine mögliche Erklärung ist, dass die Probe eventuell nicht perfekt in der X-/Y-Ebene ausgerichtet war.
Eine andere Erklärung ist, dass die Strahlbreite $d_0$ und die Probenlänge $D$ nicht exakt genug gemessen wurden.

In der Messung der Reflektivität (\autoref{fig:plot_messung}) sind die Kiessig-Oszillationen klar zu erkennen.
Die Kurve der idealen Fresnelreflektivität von Si stimmt aber nur in der Größenordnung mit der Messung überein.
Der Parratt-Algorithmus lies sich ab einem bestimmten Winkel gut für die Messung verwenden, 
aber die Parameter ließen sich nicht so anpassen, dass die ganze Kurve mit den Messdaten übereinstimmt.

Die ermittelten Schichtdicken der Polystyrolschicht
\begin{align*}
    d_\text{PS,Minima} &= \SI{8.82+-0.31e-8}{\meter} \\
    d_\text{PS,Parratt} &= \SI{8.55e-8}{\meter}
\end{align*}
stimmen miteinander überein und die wahre Schichtdicke scheint in diesem Bereich zu liegen.

Die ermittelten Dispersionen und die Literaturwerte sind
\begin{align*}
    \delta_\text{PS,gemessen} &= \num{0.5e-6} \\
    \delta_\text{PS,Literatur} &= \num{3.5e-6} \\
    \delta_\text{Si,gemessen} &= \num{6.75e-6} \\
    \delta_\text{Si,Literatur} &= \num{7.6e-6}
\end{align*}
und sie stimmen in der Größenordnung überein, aber zeigen deutliche Abweichungen.\cite{V44old}
Die Dispersion von Silizium konnte dabei genauer gemessen werden.
Die geringe gemessene Dispersion von Polystyrol ist darauf zurückzuführen, 
dass die Kiessig-Oszillationen eine sehr geringe Amplitude gezeigt haben.
Woher diese Amplitudenabschwächung kommt, bleibt aber unklar.

Die berechneten kritischen Winkel und deren Literaturwerte sind
\begin{align*}
    \alpha_\text{c,PS,gemessen} &= \SI{0.057}{\degree} \\
    \alpha_\text{c,PS,Literatur} &= \SI{0.153}{\degree} \\
    \alpha_\text{c,Si,gemessen} &= \SI{0.211}{\degree} \\
    \alpha_\text{c,Si,Literatur} &= \SI{0.223}{\degree}
\end{align*} 
und zeigen leichte Abweichungen.
Die gemessenen kritischen Winkel zeigen in \autoref{fig:plot_messung} eine gute Übereinstimmung mit den Einbrüchen in der Intensität.
Besonders der kritische Winkel von Polystyrol ist nicht gut bestimmt worden.
Da $\alpha_\text{c}$ aber aus $\delta$ berechnet wurde, ist dies nicht verwunderlich.

Die ermittelten Rauigkeiten
\begin{align*}
    \sigma_\text{Luft,PS} &= \SI{8.0e-10}{\meter} \\
    \sigma_\text{PS,Si} &= \SI{6.3e-10}{\meter}
\end{align*}
können mit keinem Wert verglichen werden.
Allerdings ist anzunehmen, dass $\sigma_\text{PS,Si}$ recht genau sein sollte, 
da dieser Parameter einen großen Einfluss auf die Kurve des Parratt-Algorithmus hatte 
und die sonstigen Parameter von Silizium gut mit den Vergleichswerten übereinstimmen.
Der Parameter $\sigma_\text{Luft,PS}$ hatte hingegen keinen großen Einfluss auf die Kurve und konnte nicht genau bestimmt werden.

Allgemein war die Anpassung des Parratt-Algorithmus eher ungenau und es ist möglich, 
dass eine andere Zusammenstellung von Parametern eine bessere Anpassung der Theoriekurve an die Messdaten hätte.
Dass der diffuse Scan so geringe Intensitäten hatte und die Amplituden der Kiessig-Oszillationen so gering sind,
weist auf einen systematischen Fehler während der Messung hin.
Die Ursache dieses Fehlers kann allerdings nicht näher bestimmt werden.
