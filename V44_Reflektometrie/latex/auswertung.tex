\section{Auswertung}
\label{sec:Auswertung}

\subsection{Justierung des Detektors}
\label{ssec:Justierung}

Zunächst werden sowohl der Detektor als auch die Probe im Detektor durch verschiedene Scans justiert.
Hieraus ergeben sich Konstanten, die zur Auswertung der eigentlichen Röntgenreflektometrie benötigt werden.

\subsubsection{Detektorscan}
\label{sssec:Detektorscan}

Wie in der Durchführung beschrieben, wird zuerst ein Detektorscan durchgeführt, also ein winkelabhängiger Scan ohne die Probe.
Das Ergebnis dieses Scans ist in \autoref{fig:plot_detektorscan} zu sehen. 

\begin{figure}
    \centering
    \includegraphics[width=\textwidth]{build/plot_detektorscan.pdf}
    \caption{Detektorscan mit entsprechender Gaußfunktion als Ausgleichskurve}
    \label{fig:plot_detektorscan}
\end{figure}

An diesen Scan wird eine Gaußfunktion der Form
\begin{equation}
    I(\alpha) = \frac{a}{\sqrt{2 \cdot \pi \cdot \sigma^2}} \cdot \exp\left(-\frac{(x - \mu)^2}{2 \cdot \sigma^2}\right) + b
\end{equation}
mithilfe der Funktion \textit{curve\_fit} aus der Python Bibliothek SciPy angepasst.\cite{scipy}
Damit ergeben sich die Parameter 
\begin{align*}
    a &= \num{1.892+-0.038e+05} \\
    b &= \num{4.0+-1.2e+04} \\
    \sigma &= \SI{0.04671+-0.00083}{\degree} \\
    \mu &= \SI{0.04603+-0.00068}{\degree} \, .
\end{align*}

Es wird sowohl die Halbwertsbreite $FWHM$ als auch die maximale Intensität $I_\text{max}$ dieser Gaußfunktion abgelesen.
So ergeben sich
\begin{align*}
    FWHM &= \SI{0.112}{\degree} \\
    I_\text{max} &= \num{1.656e+06} \, .
\end{align*}

\subsubsection{Z-Scan}
\label{sssec:Z-Scan}

Aus dem ersten durchgeführten Z-Scan, der in \autoref{fig:plot_zscan} zu sehen ist,
wird die Strahlbreite $d_0$ als der Abstand von maximaler zu minimaler Intensität bestimmt.

\begin{figure}
    \centering
    \includegraphics[width=\textwidth]{build/plot_zscan.pdf}
    \caption{Erster durchgeführter Z-Scan mit eingezeichneter Strahlbreite}
    \label{fig:plot_zscan}
\end{figure}

Hier ergibt sich eine Strahlbreite von
\begin{equation*}
    d_0 = \SI{0.22}{\milli\meter} \, .
\end{equation*}

\subsubsection{Rockingscan}
\label{sssec:Rockingscan}

Aus dem ersten durchgeführten Rockingscan, der in \autoref{fig:plot_rockingscan} zu sehen ist,
wird der Geometriewinkel $\alpha_\text{g}$ bestimmt.

\begin{figure}
    \centering
    \includegraphics[width=\textwidth]{build/plot_rockingscan.pdf}
    \caption{Erster durchgeführter Rockingscan mit eingezeichneten Geometriewinkeln}
    \label{fig:plot_rockingscan}
\end{figure}

Es können zwei Geometriewinkel abgelesen werden und diese werden zu einem Wert gemittelt.
So ergeben sich
\begin{align*}
    \alpha_\text{g,links} &= \SI{0.65}{\degree} \\
    \alpha_\text{g,rechts} &= \SI{0.75}{\degree} \\
    \overline{\alpha_\text{g}} &= \SI{0.70}{\degree} \, .
\end{align*}

Außerdem kann aus der Strahlbreite $d_0 = \SI{0.22}{\milli\meter}$ und der Probenlänge $D = \SI{20}{\milli\meter}$
der Geometriewinkel über \autoref{eq:a_g} zu
\begin{equation*}
    \alpha_\text{g,Theorie} = \SI{0.63}{\degree}
\end{equation*}
berechnet werden. 

\subsection{Messung des Reflektivitätsscans und Bestimmung des Dispersionsprofils}
\label{ssec:Messung}

Anhand des aufgenommenen Reflektivitätsscans wird nun das Dispersionsprofil der untersuchten Probe bestimmt.
Die Messdaten und die Ergebnisse der hier durchgeführten Berechnungen sind in \autoref{fig:plot_messung} zu sehen.

\begin{figure}
    \centering
    \includegraphics[width=\textwidth]{build/plot_messung.pdf}
    \caption{Messdaten des Reflektivitätsscans und Ergebnisse der Bestimmung des Dispersionsprofils. 
    Die Messdaten sind durch 10 geteilt, damit nicht so viele Linien übereinander liegen.} 
    \label{fig:plot_messung}
\end{figure}

Die Auswertung des Reflektivitätsscans wird auf einen Bereich von $\alpha_\text{i} \in [\SI{0.01}{\degree}, \SI{1.6}{\degree}]$ beschränkt,
da die Werte außerhalb dieses Bereichs unbrauchbar für die Auswertung sind.

Anstatt der gemessenen Intensität $I$ (Hits pro 5 Sekunden) wird mit der Reflektivität $R$ gerechnet.
\begin{equation}
    R = \frac{I}{5 \cdot I_\text{max}}
\end{equation}
ist also die Intensität normiert auf das fünffache der maximalen Intensität des Detektorscans während der Justierung.

Um Rückstreueffekte zu eliminieren wird der diffuse Scan vom Reflektivitätsscan abgezogen. 
Hierbei ist anzumerken, dass der von uns gemessenene diffuse Scan sehr niedrige Intensitäten hatte, 
somit ist in \autoref{fig:plot_messung} kaum ein Unterschied in den korrigierten Reflektivitäten zu sehen.

Mithilfe des während der Justierung bestimmten Geometriewinkels $\alpha_\text{g}$, der Strahlbreite $d_0$ sowie der Probenlänge $D$
wird nun nach \autoref{eq:geo} der Geometriefaktor für jeden Winkel $\alpha_\text{i}$ bestimmt.
Die Reflektivität wird mit diesen Geometriefaktoren korrigiert.

Zum Vergleich mit der Theorie wird außerdem in \autoref{fig:plot_messung} die Fresnelreflektivität einer ideal glatten Siliziumoberfläche gezeichnet.
Diese wird über \autoref{eq:reflek} und dem Literaturwert $\alpha_\text{c,Si} = \SI{0.223}{\degree}$ berechnet.\cite{V44old}

Um aus dem korrigierten Reflektivitätsscan die Schichtdicke der Polystyrolschicht der Probe zu bestimmen,
werden die Abstände $\Delta\alpha_\text{i}$ der Minima der Kiessig-Oszillation gemessen.
Dies wird mithilfe der Funktion \textit{find\_peaks} aus der Python Bibliothek SciPy getan
und die Minima werden auch in \autoref{fig:plot_messung} eingezeichnet.\cite{scipy}
Anschließend wird der Mittelwert dieser Abstände zu
\begin{equation*}
    \overline{\Delta\alpha_\text{i}} = \SI{8.73+-0.31e-4}{\,\degree}
\end{equation*}
berechnet.
Über \autoref{eq:schicht} wird die Schichtdicke zu 
\begin{equation*}
    d_\text{PS} = \SI{8.82+-0.31e-08}{\meter}
\end{equation*}
berechnet.

Eine andere Möglichkeit die Schichtidicke der Polystyrolschicht zu ermitteln ist der Parratt-Algorithmus.
Hiermit lässt sich das Dispersionsprofil der Probe bestimmen.
Dieser wird genutzt indem über \autoref{eq:parratt},\ref{eq:k} und \ref{eq:r_modifiziert} für jeden Einfallswinkel $\alpha_\text{i}$ die Refletivität berechnet wird.
Wobei hier zwei Schichten betrachtet werden, Polystyrol (PS) als Schicht 2 und Silizium (Si) als Substrat bzw. Schicht 3.
Diese berechnete Reflektivität wird in einem Plot mit der korrigierten gemessenen Reflektivität verglichen.
Die Parameter $\delta_2,\delta_3,\sigma_1,\sigma_2$ und $z_2$ werden manuell so angepasst,
dass die berechnete Reflektivität möglichst gut mit der gemessenen Reflektivität übereinstimmt.

Als bereits bekannte Parameter werden 
\begin{align*}
    \delta_\text{1,Luft} &= 0 \\
    z_1 &= 0 \\
    \lambda &= \SI{1.54e-10}{\meter}
\end{align*}
angenommen. 
Die in \autoref{fig:plot_messung} gezeigte Theoriekurve ensteht mit den Parametern
\begin{align*}
    \delta_\text{2,PS} &= \num{0.5e-6} \\
    \delta_\text{3,Si} &= \num{6.75e-6} \\
    \sigma_1 &= \SI{8.0e-10}{\meter} \\
    \sigma_2 &= \SI{6.3e-10}{\meter} \\
    z_2 &= \SI{8.55e-8}{\meter} \, .
\end{align*}
Allerdings muss angemerkt werden, dass $\sigma_1$ nur wenig Einfluss auf die Gestalt der berechneten Reflektivitätskurve hat.
Außerdem sind diese Parameter aus subjektiver Wahrnehmung entstanden.

Mithilfe der zuvor bestimmten Dispersionen $\delta_\text{PS}$ und $\delta_\text{Si}$ 
lassen sich nun die kritischen Winkel $\alpha_\text{c}$ über \autoref{eq:a_c} zu
\begin{align*}
    \alpha_\text{c,PS} &= \SI{0.057}{\degree} \\
    \alpha_\text{c,Si} &= \SI{0.211}{\degree}
\end{align*}
bestimmen. 
Auch diese Winkel sind in \autoref{fig:plot_messung} eingezeichnet.


