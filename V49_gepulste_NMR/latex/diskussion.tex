\section{Diskussion}
\label{sec:Diskussion}

\subsection{Vergleich von $T_1$}

Zunächst wird der von uns berechnete Wert für die Spin‐Gitter‐Relaxationszeit $T_1$ mit dem Literaturwert $T_{1,\text{lit}}$ verglichen. \cite{T1}
Die Werte sind dabei 
\begin{align*}
    T_1 =& \SI{1.93(12)}{\second} \\
    T_{1,\text{lit}} =& \SI{3.15}{\second} \, .
\end{align*}
Der Literaturwert entspricht der Relaxationszeit bei einer Temperatur von $T = \SI{20}{\celsius}$, da zu Beginn des Versuches die Temperatur 
$T = \SI{22.1}{\celsius}$ betrug.
Die Abweichung von einander ist somit
\begin{equation}
    \Delta T_1 = \num{34.92} \% \, .
\end{equation}
Das ist eine relativ hohe Abweichung.
Betrachtet man \autoref{fig:t1}, den Plot der zur Auswertung verwendet wurde, wird auch teilweise klar warum.
Zu Beginn liegen einige der Messwerte nicht wirklich auf der Fit-Funktion.
Das liegt vor allem daran, dass die Messwerte kurzzeitig sinken, obwohl das eigentlich nicht passieren dürfte.
Es ist davon auszugehen, dass hier etwas nicht korrekt funktioniert hat. 
Diese Werte wurden von einem Oszilloskop abgelesen, weswegen ein systematischer Fehler aufgetreten sein könnte.

\subsection{Diffusionskoeffizient und Molekülradius}

Der Diffusionskoeffizient wird ebenfalls mit einem Literaturwert vergleichen. 
Da diese Messung gegen Ende stattfand, wird hier der Wert für $T = \SI{25}{\celsius}$ verwendet. \cite{D}
Die Messwerte für $D$ sind dann
\begin{align*}
    D =& \SI{2.533(17)e-9}{\meter\squared\per\second} \\
    D_\text{lit} =& \SI{2.299e-9}{\meter\squared\per\second} \, .
\end{align*}
Damit erhalten wir eine Abweichung von 
\begin{equation}
    \Delta D = \num{8.62} \% \, .
\end{equation}
Vergleichen mit der vorherigen Abweichung erscheint diese wesentlich genauer zu sein. 
Das liegt aber auch vor allem daran, dass wir einige sehr schlechte Werte entfernt haben, wie bereits erwähnt. 
So konnten die Parameter des Fits sehr genau bestimmt werden.

Die letzte Überprüfung der Genauigkeit dieser Messung findet bei der Bestimmung der Molekülradien statt.
Beide möglichen Werte wurden bereits in der Auswertung bestimmt. 
Daher wird hier nur die Abweichung 
\begin{equation}
    \Delta r = \num{50.46} \% \, .
\end{equation}
berechnet.
Die Abweichung ist offensichtlich recht groß.
Einerseits war $D$ natürlich eine fehlerbehaftete Größe und somit nicht exakt bestimmt.
Zudem wurde hier eine Temperatur angegeben, die erst nach der eigentlichen Messung bestimmt worden war.
Vor allem war die Viskosität $\eta$ aber nicht exakt an die Temperatur angepasst. 
Die Viskosität ändert sich mit der Temperatur und wir konnten die Viskositätsmessung nicht durchführen.