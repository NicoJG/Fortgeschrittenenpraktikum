\section{Auswertung}
\label{sec:Auswertung}

Alle hier angegebenen Unsicherheiten wurden mit der Python Bibliothek Uncertainties berechnet.\cite{uncertainties}
Diese Bibliothek basiert auf der Gauß'schen Fehlerfortpflanzung
\begin{equation}
    \Delta y = \sum_{i=1}^n \left| \frac{\delta f(x_1,...,x_n)}{\delta x_i} \right| \Delta x_i \, .
    \label{eq:fehlerrechnung}
\end{equation}

Alle Ausgleichsrechnungen wurden mit der Funktion \textit{curve\_fit} aus der Python Bibliothek SciPy durchgeführt. \cite{scipy}

Ziel dieser Auswertung ist es mit Hilfe der gepulsten Kernspinresonanz den Diffusionskoeffizienten $D$ zu berechnen. 
Damit diese Berechnung gelingt werden zunächst die beiden Relaxationszeiten $T_1$ und $T_2$ berechnet.
Als letzte Überprüfung der Auswertung wird der Molekülradius von Wasser über die gemessenen Werte bestimmt.

\subsection{Bestimmung von \texorpdfstring{$T_1$}{T1}}
\label{ssec:aus1}

Für die Berechnung von $T_1$ wird zunächst ein $\SI{180}{\degree}$-Puls durchgeführt.
Nach einer Zeitspanne $\tau$ folgt ein $\SI{90}{\degree}$-Puls. 
Die danach gemessenen Spannungen $U$ sind für verschiedene Werte $\tau$ in \autoref{tab:t1} notiert. 

\begin{table}
    \centering
    \caption{Gemessene Spannungen in Abhängigkeit von $\tau$ für die $T_1$ Bestimmung}
    \label{tab:t1}
    \begin{tabular}{S[table-format=5.1] S[table-format=-4.0]}
        \toprule
        \tableSI{\tau}{\milli\second} & \tableSI{U}{\milli\volt}  \\
        \midrule
        1.0 &  -1075  \\
        1.6 & -1013 \\
        2.6 & -994 \\
        4.3 & -1000 \\
        7.0 & -1075 \\
        11.3 & -1037 \\
        18.3 & -1012 \\
        29.8 & -998 \\
        48.3 & -956 \\
        78.5 & -931 \\
        127.0 & -888 \\
        \bottomrule
    \end{tabular}
    \begin{tabular}{S[table-format=5.1] S[table-format=-4.0]}
        \toprule
        \tableSI{\tau}{\milli\second} & \tableSI{U}{\milli\volt}  \\
        \midrule
        207.0 & -799 \\
        336.0 & -708 \\
        546.0 & -580 \\
        886.0 & -415 \\
        1440.0 & -201 \\
        2340.0 & 66 \\
        3000.0 & 117 \\
        3800.0 & 309 \\
        6160.0 & 508 \\
        10000.0 & 545 \\
        & \\
        \bottomrule
    \end{tabular}
\end{table}

Anschließend werden diese Werte nach \autoref{eq:T1_relaxation} mit der Fit-Funktion
\begin{equation}
    U(\tau) = a \cdot \exp\left(-\frac{\tau}{T_1}\right) + c 
    \label{eq:fit_t1}
\end{equation}
gefittet.
Dabei ergeben sich die Parameter 
\begin{align*}
    a &= \SI{-1.55(03)}{\volt} \\
    T_1 &= \SI{1.93(12)}{\second} \\
    c &=  \SI{0.53(03)}{\volt} \, .
\end{align*}
Die Werte aus \autoref{tab:t1} sind nachfolgend aufgetragen.
$T_1$ ist damit direkt aus dem Fit als 
\begin{equation*}
    T_1 = \SI{1.93(12)}{\second}
\end{equation*}
bestimmt.

\begin{figure}
    \centering
    \includegraphics[width=\textwidth]{build/plot_T1.pdf}
    \caption{Plot der Messwerte zur Messung von $T_1$ mit exponentiellem Fit und logarithmischer X-Achse}
    \label{fig:t1}
\end{figure}

\subsection{Bestimmung von \texorpdfstring{$T_2$}{T2}}
\label{ssec:aus2}

Die Messdaten des Oszilloskops wurden bei $\tau = \SI{11.5}{\milli\second}$ aufgenommen.
Allerdings sind hier nur die Maxima der Oszillazion für die Auswertung relevant.
Durch diese wird eine Fit-Funktion nach \autoref{eq:T2_relaxation}
\begin{equation}
    U(t) = a \cdot \exp\left(-\frac{t}{T_2}\right) + c 
    \label{eq:fit_t2}
\end{equation}
gelegt.
Für diesen Fit werden die Parameter 
\begin{align*}
    a &= \SI{0.91(04)}{\volt} \\
    T_2 &= \SI{1.70(16)}{\second} \\
    c &= \SI{0.08(05)}{\volt} \, .
\end{align*}
bestimmt.
Die geplotteten Werte sind dann zusammen mit der Fit-Funktion in \autoref{fig:t2} dargestellt.
Auch hier ist die gesuchte Größe einer der Parameter, sodass sich 
\begin{equation*}
    T_2 = \SI{1.70(16)}{\second} 
    \label{eq:t2_wert}
\end{equation*}
ergibt.

\begin{figure}
    \centering
    \includegraphics[width=\textwidth]{build/plot_T2.pdf}
    \caption{Plot der Messwerte zur Messung von $T_2$ mit exponentiellem Fit}
    \label{fig:t2}
\end{figure}

Es wurden ebenfalls Oszilloskop-Daten genommen, bei denen die Einstellung \enquote{MG} auf \enquote{off} stand.
Das bedeutet, die Messung wurde nicht mit der Meiboom-Gill-Methode durchgeführt.
Dieser Plot ist in \autoref{fig:mgoff} zu sehen.
Eine Bestimmung von $T_2$ ist aus diesen Messdaten nicht möglich.

\begin{figure}
    \centering
    \includegraphics[width=\textwidth]{build/plot_mgoff.pdf}
    \caption{Messdaten des Oszilloskops ohne Meiboom-Gill-Methode}
    \label{fig:mgoff}
\end{figure}

\subsection{Bestimmung des Diffusionskoeffizienten \texorpdfstring{$D$}{D}}
\label{ssec:aus3}

Zur erfolgreichen Berechnung von $D$ über \autoref{eq:TD} werden die Gradiantenstärke $G$ und die Zeitkonstante $T_\text{D}$ benötigt. 
Diese beiden Größen werden im Folgenden bestimmt.

Für die Berechnung der Gradiantenstärke $G$ wird das Oszilloskopbild des Echos zunächst in \autoref{fig:echo} dargestellt.
Die eigentliche Bestimmung kann aber erst nach einer Fouriertransformation stattfinden. 
Alle Werte vor dem Maximum des Realteils werden hierbei verworfen.
Die Fouriertransformierte ist \autoref{fig:fourier} stark vergrößert, da der Durchmesser $d_\text{f}$ des entstandenen Halbkreises benötigt wird.

\begin{figure}
    \centering
    \begin{subfigure}{0.4\textwidth}
        \centering
        \includegraphics[width=\textwidth]{build/plot_Diff.pdf}
        \caption{Echo bei $\tau = \SI{0.2}{\milli\second}$}
        \label{fig:echo}
    \end{subfigure}
    \begin{subfigure}{0.4\textwidth}
        \centering
        \includegraphics[width=\textwidth]{build/plot_echo_Gradient.pdf}
        \caption{Stark vergrößerte Fouriertransformation des Echos}
        \label{fig:fourier}
    \end{subfigure}
    \caption{Aufgenommenes Echo mit entsprechender Fouriertransformation zur Bestimmung der Gradiantenstärke}
    \label{fig:g_messung}
\end{figure}

Der Durchmesser des Halbkreises wird etwa als 
\begin{equation*}
    d_\text{f} = \SI{13300}{\hertz} 
    \label{eq:df}
\end{equation*}
abgelesen.
Über \autoref{eq:gradient_korrektur} kann dann $G$ 
\begin{equation}
    G = \frac{2 \pi \cdot d_\text{f}}{\gamma _\text{H} \cdot d} = \SI{0.071}{\tesla\per\meter}
    \label{eq:g_wert}
\end{equation}
berechnet werden.
Dabei sind
\begin{align*}
    \gamma _\text{H} &=  \SI{268e6}{\radian\per\second\per\tesla} \\
    d &=  \SI{4.4}{\milli\meter} \, . 
\end{align*}
$\gamma _\text{H}$ ist das gyromagnetische Verhältnis von Protonen in Wasser. \cite{physics_constants}

$T_\text{D}$ erhalten wir aus einem exponentiellen Fit durch die Messwerte aus \autoref{tab:diff}.
Hierbei ist wichtig anzumerken, dass es im Intervall $\SI{0.1}{\second}$ bis $\SI{1}{\second}$ noch weitere Werte gab.
Diese lagen allerdings sehr dicht aneinander und haben den Fit teilweise verfälscht.

\begin{table}
    \centering
    \caption{Gemessene Spannungen in Abhängigkeit von $\tau$ für die Bestimmung des Diffusionskoeffizienten}
    \label{tab:diff}
    \begin{tabular}{S[table-format=2.1] S[table-format=4.0]}
        \toprule
        \tableSI{\tau}{\milli\second} & \tableSI{U}{\milli\volt}  \\
        \midrule
        0.3 & 1275 \\
        1.0 & 1258 \\
        2.0 & 1260 \\
        4.0 & 1213 \\
        6.0 & 1103 \\
        8.0 & 925 \\
        10.0 & 698 \\
        10.0 & 639 \\
        11.0 & 571 \\
        \bottomrule
    \end{tabular}
    \begin{tabular}{S[table-format=2.1] S[table-format=4.0]}
        \toprule
        \tableSI{\tau}{\milli\second} & \tableSI{U}{\milli\volt}  \\
        \midrule
        12.0 & 461 \\
        13.0 & 356 \\
        14.0 & 257 \\
        15.0 & 194 \\
        16.0 & 127 \\
        17.0 & 93 \\
        18.0 & 65 \\
        19.0 & 50 \\
        20.0 & 46 \\
        \bottomrule
    \end{tabular}
\end{table}

Der Fit wird dabei mit der Funktion 
\begin{equation}
    U(\tau) = a \cdot \exp\left(-\frac{2 \, \tau}{T_2}\right) \, \exp\left(- 2 \, \tau ^3 \cdot b\right) + c 
    \label{eq:fit_d}
\end{equation}
durchgeführt.
Hierbei ist $T_2$ der oben berechnete Wert.
Die Parameter, die sich dabei ergeben sind 
\begin{align*}
    a &= \SI{1.236(3)}{\volt} \\
    b &= \SI{304556.538(2050836)}{\per\cubic\second} \\
    c &= \SI{0.033(2)}{\volt} \, .
\end{align*}
Es ergibt sich der Plot in \autoref{fig:diff}, wobei der Fit in SI-Basiseinheiten durchgeführt wurde.
\begin{figure}
    \centering
    \includegraphics[width=\textwidth]{build/plot_Diff3.pdf}
    \caption{Plot der Messwerte zur Messung von $D$ mit exponentiellem Fit}
    \label{fig:diff}
\end{figure}

Wird nun die Fit-Funktion mit \autoref{eq:TD_relaxation} verglichen, ergibt sich der folgende Zusammenhang,
\begin{equation}
    T_\text{D} = \frac{1}{\tau ^2 \cdot b} \, . 
    \label{eq:b_td}
\end{equation}
Damit ist es nun möglich über \autoref{eq:TD} den Diffusionskoeffizienten $D$ zu berechnen.
Wir erhalten die Formel 
\begin{equation}
    D = \frac{3 \, b}{\gamma _\text{H}^2 \, G^2 } \, . 
    \label{eq:diffusion}
\end{equation}
Die Gradientenstärke $G$ haben wir bereits bestimmt und $\gamma _\text{H}$ ist das bekannte gyromagnetische Verhältnis von Protonen.
Und so ergibt sich 
\begin{equation*}
    D = \SI{2.533(17)e-9}{\meter\squared\per\second} \, . 
    \label{eq:diffusion_wert}
\end{equation*}

\subsection{Abschließende Berechnung des Molekülradius}
\label{ssec:aus4}

Der Molekülradius von Wasser wird über die Einstein-Stokes-Formel berechnet.
Formt man \autoref{eq:Einstein_Stokes} nach dem Radius $r$ um, ergibt sich 
\begin{equation}
    r =  \frac{k_\text{B} \, T}{6 \, \pi \, \eta \, D } \, .
    \label{eq:stokes}
\end{equation}
$\eta$ ist die Viskosität von Wasser $\eta = \SI{1}{\milli\pascal\second} $. \cite{wasser}
Die Temperatur $T$ wurde vor und nach dem Versuch gemessen, diese sind 
\begin{align*}
    T_\text{vor} &= \SI{295.25}{\kelvin} \\
    T_\text{nach} &= \SI{296.45}{\kelvin} \, .
\end{align*}
Da die Bestimmung von $D$ gegen Ende des Versuches stattfand, wird hier $T_\text{nach}$ verwendet.
Dadurch ergibt sich dann für den Molekülradius von Wasser
\begin{equation*}
    r_\text{gemessen} = \SI{0.857(6)}{\angstrom}  \, .
    \label{eq:radius1}
\end{equation*}

Den Vergleichswert berechnen wir über
\begin{equation}
    r =  \sqrt[3]{ \left(  \frac{M_\text{W}}{\frac{4}{3} \, \pi \, \rho \, N_\text{A} }  \right)} \, .
    \label{eq:hexagonal}
\end{equation}
Die benötigten Konstanten sind die Molekulardichte von Wasser $\rho _\text{W} = \SI{1}{\gram\per\cubic\centi\meter}$ und die Molmasse von Wasser
$M _\text{W} = \SI{18.015}{\gram\per\mol}$. \cite{wasser}
$N_\text{A}$ ist die Avogadro-Konstante.
Der Theoriewert ist dann 
\begin{equation*}
    r_\text{Theorie} = \SI{1.742}{\angstrom}  \, .
    \label{eq:radius2}
\end{equation*}
