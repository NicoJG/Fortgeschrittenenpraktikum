\section{Diskussion}
\label{sec:Diskussion}

\begin{table}
    \centering
    \caption{Vergleich der theoretischen zu den gemessenen Landé-Faktoren}
    \label{tab:diskussion}
    \begin{tabular}{l S[table-format=1.2] S[table-format=1.3] c}
        \toprule
        Spektrallinie & $g_{i,j,\text{Theorie}}$ & $g_{i,j,\text{gemessen}}$ & Abweichung \\
        \midrule
        rot             & 1.00 & 1.086 & \SI{8.6}{\percent} \\
        blau,$\pi$      & 0.50 & 0.607 & \SI{21.4}{\percent} \\
        blau,$\sigma$   & 1.75 & 1.929 & \SI{10.2}{\percent} \\
        \bottomrule
    \end{tabular}
\end{table}

In \autoref{tab:diskussion} sieht man, dass die von uns gemessenen Werte zwar in der Nähe der Theoriewerte liegen,
aber nicht sehr genau sind.
Für den Theoriewert der blauen $\sigma$-Spektrallinie wurde der Mittelwert von 2 und $3/2$ genommen, 
da hier die beiden Linien so verschwommen waren, dass sie nicht zu unterscheiden waren.
Im Allgemeinen wurden die Spektrallinien durch die Lummer-Gehrcke-Platte nicht sehr genau definiert abgebildet
und eine große Fehlerquelle könnte eben jene Unschärfe sein.
Außerdem konnten nicht die optimalen Magnetfeldstärken erreicht werden, 
sodass teilweise die gemessenen Aufspaltungen deutlich geringer waren als erhofft.
