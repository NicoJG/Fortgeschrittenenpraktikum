\section{Auswertung}
\label{sec:Auswertung}

\subsection{Bestimmung der Magnetfeldstärke}
\label{ssec:Bestimmung der Magnetfeldstärke}

Die auf- und absteigende Messung der Magnetfeldstärke $B$ je nach Stromstärke $I$ ergibt die Werte in \autoref{tab:magnet}.

% mit \tableSI
\begin{table}
    \centering
    \caption{Messergebnisse der Eichung des Elektromagneten}
    \label{tab:magnet}
    \begin{tabular}{S[table-format=1.1] S[table-format=3.0]}
        \toprule
        \tableSI{I}{\ampere} & \tableSI{B}{\milli\tesla} \\
        \midrule
        0.0 & 6 \\
        0.5 & 52 \\
        1.0 & 97 \\
        1.5 & 142 \\
        2.0 & 188 \\
        2.5 & 233 \\
        3.0 & 280 \\
        3.5 & 322 \\
        4.0 & 365 \\
        4.5 & 404 \\
        5.0 & 439 \\
        \bottomrule
    \end{tabular}
    \begin{tabular}{S[table-format=1.1] S[table-format=3.0]}
        \toprule
        \tableSI{I}{\ampere} & \tableSI{B}{\milli\tesla} \\
        \midrule
        5.0 & 439 \\
        4.5 & 409 \\
        4.0 & 372 \\
        3.5 & 331 \\
        3.0 & 287 \\
        2.5 & 240 \\
        2.0 & 195 \\
        1.5 & 149 \\
        1.0 & 102 \\
        0.5 & 54 \\
        0.0 & 7 \\
        \bottomrule
    \end{tabular}
\end{table}

Diese Werte sind in \autoref{fig:plot_magnet} veranschaulicht. 
Ebenfalls wird eine Ausgleichsrechnung mithilfe von
\begin{equation}
    B(I) = a \cdot I^2 + b \cdot I + c
\end{equation}
und der Funktion curve\_fit aus der Python Bibliothek SciPy durchgeführt. \cite{scipy}
Dies ergibt die Parameter
\begin{align}
    a &= \SI{-2.3+-0.4}{\milli\tesla\per\ampere\squared} \\
    b &= \SI{99.6+-2.1}{{\milli\tesla\per\ampere} \\
    c &= \SI{3.7+-2.2}{\milli\tesla} \, .
\end{align}

\begin{figure}
    \centering
    \includegraphics[width=\textwidth]{build/plot_magnet.pdf}
    \caption{Plot der Messwerte und Ausgleichskurve der Magnetfeldmessung}
    \label{fig:plot_magnet}
\end{figure}


\subsection{Bestimmung der Wellenlängenaufspaltung}
\label{ssec:Bestimmung der Wellenlängenaufspaltung}

Um nun aus den aufgenommenen Fotos des Interferenzmusters der Lummer-Gehrcke-Platte 
die durch den Zeemann Effekt verursachte Wellenlängenaufspaltung zu bestimmen,
werden die Fotos auf die Pixelabstände der Linien untersucht.

Hierzu wird wie in \autoref{fig:}